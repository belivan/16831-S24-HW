\documentclass{article}
\usepackage{xcolor}
\usepackage{titleps}
\usepackage[letterpaper, margin=0.95in]{geometry}
\usepackage{url}
\usepackage{amsmath}
\usepackage{amssymb}
\usepackage{wrapfig}
\usepackage{float}
\usepackage{mathtools}
\usepackage{enumitem}
\usepackage{tabu}
\usepackage{parskip}
\usepackage{natbib}
\usepackage{listings}

\usepackage[many]{tcolorbox}
\usepackage{minted}
\setminted[python]{
	% frame=single,
	% linenos,
    xleftmargin=0.475em,
    baselinestretch=1.2,
}
% https://tex.stackexchange.com/a/569249
\setcounter{secnumdepth}{5}
\setcounter{tocdepth}{5}
\makeatletter
\newcommand\subsubsubsection{\@startsection{paragraph}{4}{\z@}{-2.5ex\@plus -1ex \@minus -.25ex}{1.25ex \@plus .25ex}{\normalfont\normalsize\bfseries}}
\newcommand\subsubsubsubsection{\@startsection{subparagraph}{5}{\z@}{-2.5ex\@plus -1ex \@minus -.25ex}{1.25ex \@plus .25ex}{\normalfont\normalsize\bfseries}}
\makeatother

\usepackage{hyperref}
\usepackage[color=red]{todonotes}
\usepackage{forest}
\definecolor{light-yellow}{HTML}{FFE5CC}

\newpagestyle{ruled}
{\sethead{CMU 16-831}{Introduction to Robot Learning }{Spring 2024}\headrule
  \setfoot{}{}{}}
\pagestyle{ruled}

\renewcommand\makeheadrule{\color{black}\rule[-.75\baselineskip]{\linewidth}{0.4pt}}
\renewcommand*\footnoterule{}

\newtcolorbox[]{answer}[1][]{
    % breakable,
    enhanced,
    nobeforeafter,
    colback=white,
    title=Your Answer,
    sidebyside align=top,
    box align=top,
    #1
}



\begin{document}

\lstset{basicstyle = \ttfamily,columns=fullflexible,
backgroundcolor = \color{light-yellow}
}

\begin{centering}
    {\Large Assignment 2: Policy Gradient} \\
    \vspace{.25cm}
    % \textbf{Due September 13, 11:59 pm} \\
\end{centering}
\vspace{0.25cm}

\textbf{Andrew ID:} \texttt{ayanovic} \\
\textbf{Collaborators:} \texttt{Write the Andrew IDs of your collaborators here (if any).}\\ 
\textbf{NOTE:} Please do \textbf{NOT} change the sizes of the answer blocks or plots.

\setcounter{section}{4}
\section{Small-Scale Experiments}

\subsection{Experiment 1 (Cartpole) -- \lbrack25 points total\rbrack}

\subsubsection{Configurations}
\begin{answer}[title=Q5.1.1,height=6cm,width=\linewidth]
% TODO 
{I used the following commands to run the experiments (same as in the assignment instructions):}
\begin{minted}
[framesep=2mm, fontsize=\scriptsize, breaklines]
{bash}
python rob831/scripts/run_hw2.py --env_name CartPole-v0 -n 100 -b 1000 \
    -dsa --exp_name q1_sb_no_rtg_dsa

python rob831/scripts/run_hw2.py --env_name CartPole-v0 -n 100 -b 1000 \
    -rtg -dsa --exp_name q1_sb_rtg_dsa

python rob831/scripts/run_hw2.py --env_name CartPole-v0 -n 100 -b 1000 \
    -rtg --exp_name q1_sb_rtg_na

python rob831/scripts/run_hw2.py --env_name CartPole-v0 -n 100 -b 5000 \
    -dsa --exp_name q1_lb_no_rtg_dsa

python rob831/scripts/run_hw2.py --env_name CartPole-v0 -n 100 -b 5000 \
    -rtg -dsa --exp_name q1_lb_rtg_dsa

python rob831/scripts/run_hw2.py --env_name CartPole-v0 -n 100 -b 5000 \
    -rtg --exp_name q1_lb_rtg_na
\end{minted}
\end{answer}

\subsubsection{Plots}

\subsubsubsection{Small batch -- \lbrack5 points\rbrack}
\begin{answer}[title=Q5.1.2.1,height=9.5cm,width=\linewidth]
% TODO
\centering
\includegraphics[height=8cm]{q1_1.png}
\end{answer}

\subsubsubsection{Large batch -- \lbrack5 points\rbrack}
\begin{answer}[title=Q5.1.2.2,height=9.5cm,width=\linewidth]
% TODO
\centering
\includegraphics[height=8cm]{q1_2.png}
\end{answer}

\subsubsection{Analysis}

\subsubsubsection{Value estimator -- \lbrack5 points\rbrack}
\begin{answer}[title=Q5.1.3.1,height=4cm,width=\linewidth]
% TODO
{The reward-to-go estimator performs better than the non-reward-to-go estimator.}
\end{answer}

\subsubsubsection{Advantage standardization -- \lbrack5 points\rbrack}
\begin{answer}[title=Q5.1.3.2,height=4cm,width=\linewidth]
% TODO
{The advantage standardization performs better than the non-standardized advantage.}
\end{answer}

\subsubsubsection{Batch size -- \lbrack5 points\rbrack}
\begin{answer}[title=Q5.1.3.3,height=4cm,width=\linewidth]
% TODO
{The larger batch size appears to perform better by reducing average variance of the large-batch runs.}
\end{answer}

\subsection{Experiment 2 (InvertedPendulum) -- \lbrack15 points total\rbrack}

\subsubsection{Configurations -- \lbrack5 points\rbrack}
\begin{answer}[title=Q5.2.1,height=10cm,width=\linewidth]
% TODO
\begin{minted}
[framesep=2mm, fontsize=\scriptsize, breaklines]
{bash}
python rob831/scripts/run_hw2.py --env_name InvertedPendulum-v4 \
    --ep_len 1000 --discount 0.9 -n 100 -l 2 -s 64 -b <b*> -lr <r*> -rtg \
    --exp_name q2_b<b*>_r<r*>
python rob831/scripts/run_hw2.py --env_name InvertedPendulum-v4 \
    --ep_len 1000 --discount 0.9 -n 100 -l 2 -s 64 -b 1000 -lr 0.001 \
    --exp_name q2_b1000_r0.001
python rob831/scripts/run_hw2.py --env_name InvertedPendulum-v4 \
    --ep_len 1000 --discount 0.9 -n 100 -l 2 -s 64 -b 3000 -lr 0.005 \
    --exp_name q2_b3000_r0.005
python rob831/scripts/run_hw2.py --env_name InvertedPendulum-v4 \
    --ep_len 1000 --discount 0.9 -n 100 -l 2 -s 64 -b 3000 -lr 0.01 \
    --exp_name q2_b3000_r0.01
python rob831/scripts/run_hw2.py --env_name InvertedPendulum-v4 \
    --ep_len 1000 --discount 0.9 -n 100 -l 2 -s 64 -b 3000 -lr 0.05 \
    --exp_name q2_b3000_r0.05
python rob831/scripts/run_hw2.py --env_name InvertedPendulum-v4 \
    --ep_len 1000 --discount 0.9 -n 100 -l 2 -s 64 -b 2000 -lr 0.007 \
    --exp_name q2_b2000_r0.007
python rob831/scripts/run_hw2.py --env_name InvertedPendulum-v4 \
    --ep_len 1000 --discount 0.9 -n 100 -l 2 -s 64 -b 1700 -lr 0.007 \
    --exp_name q2_b1700_r0.007
python rob831/scripts/run_hw2.py --env_name InvertedPendulum-v4 \
    --ep_len 1000 --discount 0.9 -n 100 -l 2 -s 64 -b 1700 -lr 0.009 \
    --exp_name q2_b1700_r0.009
python rob831/scripts/run_hw2.py --env_name InvertedPendulum-v4 \
    --ep_len 1000 --discount 0.9 -n 100 -l 2 -s 64 -b 1700 -lr 0.01 \
    --exp_name q2_b1700_r0.01
\end{minted}
\end{answer}

\subsubsection{smallest \textbf{b*} and largest \textbf{r*} (same run) -- \lbrack5 points\rbrack}
\begin{answer}[title=Q5.2.2,height=4cm,width=\linewidth]
% TODO 
{batch size=1700, learning rate=0.01
Note: Although the graph shows sudden drops in performance, this training was able to reach the maximum reward of 1000.}
\end{answer}

\subsubsection{Plot -- \lbrack5 points\rbrack}
\begin{answer}[title=Q5.2.3,height=10cm,width=\linewidth]
% TODO
\centering
\includegraphics[height=8cm]{q2.png}
\end{answer}

\setcounter{section}{6}
\section{More Complex Experiments}

\subsection{Experiment 3 (LunarLander) -- \lbrack10 points total\rbrack}

\subsubsection{Configurations}
\begin{answer}[title=Q7.1.1,height=6cm,width=\linewidth]
\begin{minted}
[framesep=2mm, fontsize=\scriptsize, breaklines]
{bash}
python rob831/scripts/run_hw2.py \
    --env_name LunarLanderContinuous-v4 --ep_len 1000
    --discount 0.99 -n 100 -l 2 -s 64 -b 10000 -lr 0.005 \
    --reward_to_go --nn_baseline --exp_name q3_b10000_r0.005
\end{minted}
\end{answer}

\subsubsection{Plot -- \lbrack10 points\rbrack}
\begin{answer}[title=Q7.1.2,height=10cm,width=\linewidth]
% TODO
\centering
\includegraphics[height=8cm]{q3.png}
\end{answer}

\subsection{Experiment 4 (HalfCheetah) -- \lbrack30 points\rbrack}

\subsubsection{Configurations}
\begin{answer}[title=Q7.2.1,height=10cm,width=\linewidth]
\begin{minted}
[framesep=2mm, fontsize=\scriptsize, breaklines, escapeinside=||, mathescape=true]
{python}
python rob831/scripts/run_hw2.py --env_name HalfCheetah-v4 --ep_len 150 \
    --discount 0.95 -n 100 -l 2 -s 32 -b 10000 -lr 0.02 \
    --exp_name q4_search_b10000_lr0.02
python rob831/scripts/run_hw2.py --env_name HalfCheetah-v4 --ep_len 150 \
    --discount 0.95 -n 100 -l 2 -s 32 -b 10000 -lr 0.02 -rtg \
    --exp_name q4_search_b10000_lr0.02_rtg
python rob831/scripts/run_hw2.py --env_name HalfCheetah-v4 --ep_len 150 \
    --discount 0.95 -n 100 -l 2 -s 32 -b 10000 -lr 0.02 --nn_baseline \
    --exp_name q4_search_b10000_lr0.02_nnbaseline
python rob831/scripts/run_hw2.py --env_name HalfCheetah-v4 --ep_len 150 \
    --discount 0.95 -n 100 -l 2 -s 32 -b 10000 -lr 0.02 -rtg --nn_baseline \
    --exp_name q4_search_b10000_lr0.02_rtg_nnbaseline
\end{minted}
\end{answer}

\subsubsection{Plot -- \lbrack10 points\rbrack}
\begin{answer}[title=Q7.2.2,height=10cm,width=\linewidth]
% TODO
\centering
\includegraphics[height=8cm]{q4.png}
\end{answer}

\subsubsection{(Optional) Optimal b* and r* -- \lbrack3 points\rbrack}
\begin{answer}[title=Q7.2.3,height=4cm,width=\linewidth]
% TODO
\end{answer}

\subsubsection{(Optional) Plot -- \lbrack10 points\rbrack}
\begin{answer}[title=Q7.2.4,height=10cm,width=\linewidth]
% TODO
\centering
\includegraphics[height=8cm]{example-image-a}
\end{answer}

\subsubsection{(Optional) Describe how b* and r* affect task performance -- \lbrack7 points\rbrack}
\begin{answer}[title=Q7.2.5,height=4cm,width=\linewidth]
% TODO
\end{answer}

\subsubsection{(Optional) Configurations with optimal b* and r* -- \lbrack3 points\rbrack}
\begin{answer}[title=Q7.2.6,height=6cm,width=\linewidth]
% TODO
\begin{minted}
[framesep=2mm, fontsize=\scriptsize, breaklines]
{bash}
python rob831/scripts/run_hw2.py --env_name HalfCheetah-v4 --ep_len 150 \
    --discount 0.95 -n 100 -l 2 -s 32 -b <b*> -lr <r*> \
    --exp_name q4_b<b*>_r<r*>

python rob831/scripts/run_hw2.py --env_name HalfCheetah-v4 --ep_len 150 \
    --discount 0.95 -n 100 -l 2 -s 32 -b <b*> -lr <r*> -rtg \
    --exp_name q4_b<b*>_r<r*>_rtg

python rob831/scripts/run_hw2.py --env_name HalfCheetah-v4 --ep_len 150 \
    --discount 0.95 -n 100 -l 2 -s 32 -b <b*> -lr <r*> --nn_baseline \
    --exp_name q4_b<b*>_r<r*>_nnbaseline

python rob831/scripts/run_hw2.py --env_name HalfCheetah-v4 --ep_len 150 \
    --discount 0.95 -n 100 -l 2 -s 32 -b <b*> -lr <r*> -rtg --nn_baseline \
    --exp_name q4_b<b*>_r<r*>_rtg_nnbaseline
\end{minted}
\end{answer}

\subsubsection{(Optional) Plot for four runs with optimal b* and r* -- \lbrack7 points\rbrack}
\begin{answer}[title=Q7.2.7,height=10cm,width=\linewidth]
% TODO
\centering
\includegraphics[height=8cm]{example-image-a}
\end{answer}

\section{Implementing Generalized Advantage Estimation}

\subsection{Experiment 5 (Hopper) -- \lbrack20 points\rbrack}

\subsubsection{Configurations}
\begin{answer}[title=Q8.1.1,height=4cm,width=\linewidth]
\begin{minted}
[framesep=2mm, fontsize=\scriptsize, breaklines, escapeinside=||, mathescape=true]
{python}
# $\lambda \in [0,0.95,0.99,1]$
python rob831/scripts/run_hw2.py \
    --env_name Hopper-v4 --ep_len 1000
    --discount 0.99 -n 300 -l 2 -s 32 -b 2000 -lr 0.001 \
    --reward_to_go --nn_baseline --action_noise_std 0.5 --gae_lambda <|$\lambda$|> \
    --exp_name q5_b2000_r0.001_lambda<|$\lambda$|>
\end{minted}
\end{answer}

\subsubsection{Plot -- \lbrack13 points\rbrack}
\begin{answer}[title=Q8.1.2,height=10cm,width=\linewidth]
% TODO
\centering
\includegraphics[height=8cm]{q5.png}
\end{answer}

\subsubsection{Describe how $\lambda$ affects task performance -- \lbrack7 points\rbrack}
\begin{answer}[title=Q8.1.3,height=4cm,width=\linewidth]
% TODO
{The $\lambda$ value affects the task performance and variance of the return. 
At $\lambda=0$, the return is the lowest, and does not reach good performance. 
For $\lambda=0.95$, the return is higher, but the variance is also higher. 
For $\lambda=0.99$, the return is highest, and the variance is lower on average compared to $\lambda=0.95$. 
For $\lambda=1$, the return is slightly lower than $\lambda=0.99$, however the variance is the lowest, which is unexpected as
it is supposed to be equivalent to the vanilla neural network baseline estimator without any variance reduction. 
Moreover, when disabling the GAE estimator, the return is equivalent on average to $\lambda=1$.}
\end{answer}

\clearpage

\section{Bonus! (optional)}

\subsection{Parallelization -- \lbrack15 points\rbrack}
\begin{answer}[title=Q9.1,height=4cm,width=\linewidth]
% TODO (optional)
Difference in training time: 
\vspace{1.0cm}
\begin{minted}
[framesep=2mm, fontsize=\scriptsize, breaklines]
{bash}
python rob831/scripts/run_hw2.py \
\end{minted}
\end{answer}

\subsection{Multiple gradient steps -- \lbrack5 points\rbrack}
\begin{answer}[title=Q9.1,height=14cm,width=\linewidth]
% TODO (optional)
\centering
\includegraphics[height=8cm]{example-image-a}

\vspace{1.0cm}
\begin{minted}
[framesep=2mm, fontsize=\scriptsize, breaklines]
{bash}
python rob831/scripts/run_hw2.py \
\end{minted}

\end{answer}

\end{document}

